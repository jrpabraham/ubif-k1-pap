\documentclass[11pt, a4paper]{article}\usepackage[]{graphicx}\usepackage[]{color}
%% maxwidth is the original width if it is less than linewidth
%% otherwise use linewidth (to make sure the graphics do not exceed the margin)
\makeatletter
\def\maxwidth{ %
  \ifdim\Gin@nat@width>\linewidth
    \linewidth
  \else
    \Gin@nat@width
  \fi
}
\makeatother

\definecolor{fgcolor}{rgb}{0.345, 0.345, 0.345}
\newcommand{\hlnum}[1]{\textcolor[rgb]{0.686,0.059,0.569}{#1}}%
\newcommand{\hlstr}[1]{\textcolor[rgb]{0.192,0.494,0.8}{#1}}%
\newcommand{\hlcom}[1]{\textcolor[rgb]{0.678,0.584,0.686}{\textit{#1}}}%
\newcommand{\hlopt}[1]{\textcolor[rgb]{0,0,0}{#1}}%
\newcommand{\hlstd}[1]{\textcolor[rgb]{0.345,0.345,0.345}{#1}}%
\newcommand{\hlkwa}[1]{\textcolor[rgb]{0.161,0.373,0.58}{\textbf{#1}}}%
\newcommand{\hlkwb}[1]{\textcolor[rgb]{0.69,0.353,0.396}{#1}}%
\newcommand{\hlkwc}[1]{\textcolor[rgb]{0.333,0.667,0.333}{#1}}%
\newcommand{\hlkwd}[1]{\textcolor[rgb]{0.737,0.353,0.396}{\textbf{#1}}}%
\let\hlipl\hlkwb

\usepackage{framed}
\makeatletter
\newenvironment{kframe}{%
 \def\at@end@of@kframe{}%
 \ifinner\ifhmode%
  \def\at@end@of@kframe{\end{minipage}}%
  \begin{minipage}{\columnwidth}%
 \fi\fi%
 \def\FrameCommand##1{\hskip\@totalleftmargin \hskip-\fboxsep
 \colorbox{shadecolor}{##1}\hskip-\fboxsep
     % There is no \\@totalrightmargin, so:
     \hskip-\linewidth \hskip-\@totalleftmargin \hskip\columnwidth}%
 \MakeFramed {\advance\hsize-\width
   \@totalleftmargin\z@ \linewidth\hsize
   \@setminipage}}%
 {\par\unskip\endMakeFramed%
 \at@end@of@kframe}
\makeatother

\definecolor{shadecolor}{rgb}{.97, .97, .97}
\definecolor{messagecolor}{rgb}{0, 0, 0}
\definecolor{warningcolor}{rgb}{1, 0, 1}
\definecolor{errorcolor}{rgb}{1, 0, 0}
\newenvironment{knitrout}{}{} % an empty environment to be redefined in TeX

\usepackage{alltt}
\usepackage[margin=1in]{geometry}
\usepackage[utf8]{inputenc}
\usepackage[english]{babel}
\usepackage{csquotes}
\usepackage{longtable, booktabs, tabularx, threeparttable, adjustbox}
\usepackage{amsmath, amssymb, amsthm, bbm, bm}
\usepackage{secdot, sectsty}
\usepackage{hyperref}
\usepackage{pdflscape}
\usepackage{geometry}
\usepackage{placeins}
\usepackage{caption}
\usepackage{graphicx}
\usepackage{setspace}

\usepackage[backend=bibtex, style=authortitle, citestyle=authoryear-icomp, url=false]{biblatex}
\addbibresource{UBIF.bib}

\AtBeginEnvironment{quote}{\singlespacing\small}

\allsectionsfont{\rmfamily}
\sectionfont{\normalsize}
\subsectionfont{\normalfont\normalsize\selectfont\itshape}
\subsubsectionfont{\normalfont\normalsize\selectfont\itshape}

\newcommand{\specialcell}[2][c]{%
  \begin{tabular}[#1]{@{}c@{}}#2\end{tabular}}



% Avoid spaces and periods . in chunk labels and directory names; if your output is a TeX document, these characters can cause troubles (in general it is recommended to use alphabetic characters with words separated by - or _ and avoid other characters), e.g. setup-options is a good label, whereas setup.options and chunk 1 are bad; fig.path='figures/mcmc-' is a good prefix for figure output if this project is about MCMC, and fig.path='markov chain/monte carlo' is bad; non-alphanumeric characters except - and _ in figure filenames will be replaced with _ automatically
\IfFileExists{upquote.sty}{\usepackage{upquote}}{}
\begin{document}

\title{\textsc{Framing an Unconditional Cash Transfer: Pre-Analysis Plan}}
\author{Author\footnote{The Busara Center for Behavioral Economics}}

\maketitle

\begin{abstract}

    This document describes the pre-analysis plan for a randomized experiment examining the effects of framing of welfare payments on self-concept and investment behavior. In this study, we will provide unconditional cash transfers to residents of informal settlements in Nairobi and vary the way in which the transfers were framed to participants. Participants will be randomly assigned to one of three treatment groups: the transfer framed as a means toward poverty alleviation, individual empowerment, or collective support. We will then collect self-reported measures of self-efficacy, judgement, and affect and observed measures of temporal discounting and investment. This pre-analysis plan outlines our hypotheses, the schedule of experimental tasks, and our empirical strategy. In order to guarantee transparency and bind ourselves from fishing for results, we will pre-register the scripts to be used for data analysis.

\end{abstract}

\newpage

\tableofcontents

\newpage

\section{Introduction}

    Large scale social welfare interventions provide financial aid and a political, social, and psychological narrative about the recipients and the existing social system. One important question around unconditional cash transfers pertains to recipient motivation: will participants be motivated to engage in self-development given their new resources? Can particular frames of UBI differentially affect participant behavior and wellbeing? This field experiment will test the effect of framing aid payments on individual behavior and motivation. Participants will receive an unconditional cash transfer presented with different frames. Participants will then be offered the opportunity to engage in a set of tasks that will capture self-perception, motivation, and self-investment. We will explore whether recipients of a unconditional cash transfer will engage in different tasks compared to individuals who did not receive a payout. Additionally, we will explore how the frames influence task selection and motivation


\section{Research Design}

    \subsection{Sampling}

        The study will be conducted at the Busara Center for Behavioral Economics in Nairobi, Kenya, a facility specially designed for experimental studies. The Busara Center maintains an active participant pool of more than 12,000 Nairobi residents. For the present study, 450 participants who had previously signed up to be part of the Busara participant pool will be recruited from Kibera, an informal settlement in Nairobi. We will recruit participants using phone calls. They will be informed that they would be paid KES 300 for their participation and have the opportunity to earn more during the study. The sample will include males and females at least eighteen years old. [[Verify this and edit]]

    \subsection{Manipulation}

        At the outset of the survey, eligible and consenting participants will be told they are receiving an unconditional cash transfer from an

        \subsubsection{Poverty alleviation framing}

            \begin{quote}

                "As part of this project, we're working with a poverty alleviation organization. This organization is not Busara. It is a different organization.

                If you could give the aid a name to represent your needs and the organization's goal of poverty alleviation, what would it be? For example, it could be the food fund or emergency fund.

                Great. This is your [] fund.

                If other people you interact with regularly knew that you received aid from the [], how would they view you?

                How would receiving this aid affect your relationships?

                Can you describe to me in your own words the reasons why you think this organization is giving out aid to people like you?"

            \end{quote}

        \subsubsection{Individual empowerment framing}

            \begin{quote}

                "As part of this project, we're working with an individual empowerment organization. This organization is not Busara. It is a different organization.

                As a reminder, these resources are intended to help you help yourself and to choose how to live your own life.

                Can you tell me: what are your most important goals for yourself?

                If you could give the resources a name to represent your individual goals, and the organization's goal of [], what would it be?  For example, it could be your self-investment fund or your business fund.

                Great. This is your [] fund.

                How would receiving this aid affect your relationships?

                Can you describe to me in your own words the reasons why you think this organization is giving out aid to people like you?"

            \end{quote}

        \subsubsection{Collective support framing}

            \begin{quote}

                "As part of this project, we're working with a group support organization. This organization is not Busara. It is a different organization.

                As a reminder, these resources are intended to help you find ways to support and empower people in your family and in your community that you care about most.

                Can you tell me: what are the most important goals you have for helping your family and the people you care most about? Who would you be helping?

                If you could give these resources a name to represent the goals you have for helping your community and the organization's goal of [], what would it be? For example, it could be your education fund or your growing together fund.

                Great. This is your [] fund.

                If other people you interact with regularly knew that you received these resources from the [], how would they view you?

                How would receiving these resources affect your relationships?

                Can you describe to me in your own words the reasons why you think this organization is giving out resources to people like you?"

            \end{quote}

    \subsection{Data collection}

        \subsubsection{Field protocol}

            Respondent payments, including participation fees and cash transfers will be delivered using via the mobile money system M-Pesa.\footnote{For more information on M-Pesa, see Mbiti and Weil (2015) and Jack and Suri (2011).}

        \subsubsection{Survey}

\section{Empirical Analysis}

    \subsection{Treatment effect of policy frames}

        We will use the following reduced-form specification to estimate the treatment effect of policy frames.\footnote{We will conduct the data analysis outlined in this section using the R programming language with the scripts included in Appendix \ref{sec:rscripts}.}

  		\begin{equation} \label{eq:teffect}
            Y_{i} = \beta_{0} + \beta_{1}\text{\textsc{Ind}}_{i} + \beta_{2}\text{\textsc{Col}}_{i} + \varepsilon_{i}
		\end{equation}

        $Y_{i}$ refers to the outcome variables for individual $i$ measured after the manipulation. \textsc{Ind}$_{i}$ indicates assignment to the individual empowerment frame while \textsc{Col}$_{i}$ indicates assignment to the collective support frame. The reference category in this model is the poverty alleviation frame. We will estimate cluster-robust standard errors at the individual level. Table \ref{tab:hypotheses} lists the hypotheses we will test using Equation \ref{eq:teffect}.

        \begin{table}[h]
        \centering
        \caption{Primary hypothesis tests}
        \label{tab:hypotheses}
        \begin{tabular}{@{}lllll@{}}
        \toprule
        Null hypothesis                                     & Description                                          &  &  &  \\ \midrule
        $H_0: \beta_1 = 0$ & Effect of individual empowerment frame relative to poverty alleviation frame &  &  &  \\
        $H_0: \beta_2 = 0$ & Effect of collective support frame relative to poverty alleviation frame &  &  &  \\
        $H_0: \beta_1 = \beta_2$ & Effect of collective support frame relative to individual empowerment frame &  &  &  \\ \bottomrule
        \end{tabular}
        \end{table}

        To improve precision, we will also apply covariate adjustment with a vector of baseline indicators. We obtain the covariate-adjusted treatment effect estimate by estimating Equation \ref{eq:teffect} including the demeaned covariate vector $\mathbf{\dot X}_{i} = \mathbf{X}_{i} - \mathbf{\bar X}_{i}$ as an additive term and as an interaction with the treatment indicator.

        \begin{equation} \label{eq:controls}
            Y_{i} = \beta_{0} + \beta_{1}\text{\textsc{Ind}}_{i} + \beta_{2}\text{\textsc{Col}}_{i} + \gamma_{0} \mathbf{\dot X}'_i + \gamma_{1}\text{\textsc{Ind}}_{i} \mathbf{\dot X}'_i + \gamma_{2}\text{\textsc{Col}}_{i} \mathbf{\dot X}'_i + \varepsilon_{i}
        \end{equation}

        The set of indicators partitions our sample so that our estimate for $\beta_j$ remains unbiased for the average treatment effect \parencite{lin_agnostic_2013}. We will estimate cluster-robust standard errors at the individual level. We use this model to test the hypotheses detailed in Table \ref{tab:hypotheses} including the control variables listed in Table \ref{tab:controlvars}.

        \begin{table}[h]
        \centering
        \caption{Control variables for covariate adjustment}
        \label{tab:controlvars}
        \begin{tabular}{@{}lllll@{}}
        \toprule
        Variable                                     & Description                                          &  &  &  \\ \midrule
        Age & Dummy variable indicating participant is over 25 &  &  &  \\
        Gender & Dummy variable indicating participant is female &  &  &  \\
        Marital status & Dummy variable indicating participant is married or co-habitating 2 &  &  &  \\
        Education & Dummy variable indicating participant completed std. 8 &  &  &  \\
        Children & Dummy variable indicating participant has children &  &  &  \\ \bottomrule
        \end{tabular}
        \end{table}

        % If we want to be really cool, we can use machine learning to choose covariates

    \subsection{Randomization inference}

        One potential concern is that inference might be invalidated by finite sample bias in estimates of the standard errors. To address this issue, we conduct randomization inference to test the Fisherian sharp null hypothesis of no treatment effect for every participant \parencite{fisher_design_1935}.\footnote{Note that this is more restrictive than the null hypothesis of zero average treatment effect we test in the previous section.} The basis for this inferential framework is that the distribution of test statistics comes from random treatment assignment rather than from drawing a finite sample from a super-population. This method produces exact $p$-values which do not rely on asymptotic theorems for valid inference. We perform Monte Carlo approximations of the exact $p$-values using $M=10,000$ permutations of the treatment assignment. We then estimate our primary specification within each $m^{th}$ permutation and calculate the standard Wald statistics for each of our hypothesis tests. We compare the Wald statistics from the original sample with the distribution of permuted statistics to produce approximations of the exact $p$-values:

        \begin{equation} \label{eq:exactp}
            \hat{p}_{\beta} =  \frac{1}{10,000}\sum_{m=1}^{10,000} \mathbf{1} \Big [ \mathbf{\hat{\beta'}}_m V(\mathbf{\hat{\beta}}_m)^{-1} \mathbf{\hat{\beta}}_m \geq \mathbf{\hat{\beta'}}_{obs.} V(\mathbf{\hat{\beta}}_{obs.})^{-1} \hat{\beta}_{obs.} \Big ]
        \end{equation}

        Following \textcite{young_channeling_2015}, we permute the data and calculate the regressions for all outcomes within each draw.

    \subsection{Heterogeneous treatment effects}

        We will analyze the extent to which the policy frames produced heterogeneous treatment effects with the following specification.

        \begin{equation}
            Y_{i} = \beta_{0} + \beta_{1}\text{\textsc{Ind}}_{i} + \beta_{2}\text{\textsc{Col}}_{i} + \delta_{0} x_i + \delta_{1}\text{\textsc{Ind}}_{i} x_i + \delta_{2}\text{\textsc{Col}}_{i} x_i + \varepsilon_{i}
        \label{eq:heteffect} \end{equation}

        $x_{i}$ is the binary dimension of heterogeneity measured before treatment assignment. $\delta_{1}$ and $\delta_{2}$ identify the heterogeneous treatment effects of the individual empowerment and collective support frames relative to the poverty alleviation frame. Testing $\delta_{1} = \delta_{2}$ identifies heterogeneous effects between the former two frames. Standard errors are clustered at the individual level. We estimate this model with the baseline variables summarized in Table \ref{tab:hetvars}.

        \begin{table}[h]
        \centering
        \caption{Dimensions of heterogeneity}
        \label{tab:hetvars}
        \begin{tabular}{@{}lllll@{}}
        \toprule
        Variable                                     & Description                                          &  &  &  \\ \midrule
        Age & Dummy variable indicating participant is over 25 &  &  &  \\
        Gender & Dummy variable indicating participant is female &  &  &  \\
        Marital status & Dummy variable indicating participant is married or co-habitating 2 &  &  &  \\
        Education & Dummy variable indicating participant completed std. 8 &  &  &  \\
        Children & Dummy variable indicating participant has children &  &  &  \\ \bottomrule
        \end{tabular}
        \end{table}

    \subsection{Multiple testing adjustment}

        Given that our survey instrument included several items related to a single behavior or dimension, we calculate sharpened $q$-values over all outcomes following \textcite{benjamini_adaptive_2006} to control the false discovery rate (FDR). Rather than specifying a single $q$, we report the minimum $q$-value at which each hypothesis is rejected \parencite{anderson_multiple_2008}. We apply this correction separately for each hypothesis test. We report both standard $p$-values and minimum $q$-values in our analysis.

    \subsection{Outcomes of interest}

\newpage

\printbibliography

\newpage

\appendix

\section{Consent Form}
\section{Survey Instruments}
\section{Data Analysis Scripts} \label{sec:rscripts}

    \subsection{Generating data}

\begin{knitrout}
\definecolor{shadecolor}{rgb}{0.969, 0.969, 0.969}\color{fgcolor}\begin{kframe}
\begin{alltt}
  \hlcom{#Create locals for simulation}
    \hlstd{OBS} \hlkwb{<-} \hlnum{1000}
    \hlstd{TE}  \hlkwb{<-} \hlnum{.8}
    \hlstd{HET} \hlkwb{<-} \hlnum{.4}

  \hlcom{#Generate treatment}
    \hlstd{Treat} \hlkwb{<-} \hlkwd{sample}\hlstd{(}\hlnum{0}\hlopt{:}\hlnum{1}\hlstd{,OBS,}\hlkwc{rep} \hlstd{=} \hlnum{TRUE}\hlstd{,}\hlkwc{prob} \hlstd{=} \hlkwd{c}\hlstd{(}\hlnum{.5}\hlstd{,}\hlnum{.5}\hlstd{))} \hlopt
    \hlkwd{factor}\hlstd{(}\hlkwc{levels} \hlstd{=} \hlkwd{c}\hlstd{(}\hlnum{0}\hlstd{,}\hlnum{1}\hlstd{),} \hlkwc{labels} \hlstd{=} \hlkwd{c}\hlstd{(}\hlstr{"Control"}\hlstd{,}\hlstr{"Treatment"}\hlstd{))}

  \hlcom{#Generate gender}
   \hlstd{Gen} \hlkwb{<-} \hlkwd{sample}\hlstd{(}\hlnum{0}\hlopt{:}\hlnum{1}\hlstd{,OBS,}\hlkwc{rep} \hlstd{=} \hlnum{TRUE}\hlstd{,}\hlkwc{prob} \hlstd{=} \hlkwd{c}\hlstd{(}\hlnum{.5}\hlstd{,}\hlnum{.5}\hlstd{))}  \hlopt
    \hlkwd{factor}\hlstd{(}\hlkwc{levels} \hlstd{=} \hlkwd{c}\hlstd{(}\hlnum{0}\hlstd{,}\hlnum{1}\hlstd{),} \hlkwc{labels} \hlstd{=} \hlkwd{c}\hlstd{(}\hlstr{"Male"}\hlstd{,}\hlstr{"Female"}\hlstd{))}

  \hlcom{#Generate factor variable measuring highest level of education}
   \hlstd{Edu} \hlkwb{<-} \hlkwd{sample}\hlstd{(}\hlnum{1}\hlopt{:}\hlnum{3}\hlstd{,OBS,}\hlkwc{rep} \hlstd{=} \hlnum{TRUE}\hlstd{,}\hlkwc{prob} \hlstd{=} \hlkwd{c}\hlstd{(}\hlnum{.5}\hlstd{,}\hlnum{.3}\hlstd{,}\hlnum{.2}\hlstd{))} \hlopt
    \hlkwd{factor}\hlstd{(}\hlkwc{levels} \hlstd{=} \hlkwd{c}\hlstd{(}\hlnum{1}\hlstd{,}\hlnum{2}\hlstd{,}\hlnum{3}\hlstd{),} \hlkwc{labels} \hlstd{=} \hlkwd{c}\hlstd{(}\hlstr{"Primary school"}\hlstd{,}\hlstr{"High school"}\hlstd{,}\hlstr{"University & above"}\hlstd{))}

  \hlcom{#Generate income}
   \hlstd{LnInc} \hlkwb{<-} \hlkwd{rnorm}\hlstd{(OBS,} \hlkwc{mean} \hlstd{=} \hlnum{5}\hlstd{,} \hlkwc{sd} \hlstd{=} \hlnum{1}\hlstd{)}
   \hlstd{Inc} \hlkwb{<-} \hlkwd{exp}\hlstd{(LnInc)}

  \hlcom{#Generate y with notreatment effect}
    \hlstd{y_nottreat} \hlkwb{<-} \hlkwd{rnorm}\hlstd{(OBS,} \hlnum{0}\hlstd{,} \hlnum{1}\hlstd{)}

  \hlcom{#Generate outcome with noisy treatment effect of ___}
    \hlstd{y_Teffect} \hlkwb{<-} \hlkwd{rnorm}\hlstd{(OBS, TE,} \hlnum{1}\hlstd{)}
    \hlstd{y_Teffect[Treat} \hlopt{==} \hlstr{"Control"}\hlstd{]} \hlkwb{<-} \hlnum{0}
    \hlstd{y_treated} \hlkwb{=} \hlstd{y_nottreat} \hlopt{+} \hlstd{y_Teffect}

  \hlcom{#Generate outcome with noisy treatment effect of ___ and noisy het of ___ gender}
    \hlstd{y_GenTeffect} \hlkwb{<-} \hlkwd{rnorm}\hlstd{(OBS, HET,} \hlnum{1}\hlstd{)}
    \hlstd{y_GenTeffect[Treat} \hlopt{==} \hlstr{"Control"}\hlstd{]} \hlkwb{<-} \hlnum{0}
    \hlstd{y_GenTeffect[Gen} \hlopt{==} \hlstr{"Male"}\hlstd{]} \hlkwb{<-} \hlnum{0}
    \hlstd{y_HetTreated} \hlkwb{<-} \hlstd{y_treated} \hlopt{+} \hlstd{y_GenTeffect}

  \hlcom{#Create, save dataframe}
    \hlstd{SimTreat} \hlkwb{<-} \hlkwd{data.frame}\hlstd{(Treat, Gen, Edu, Inc, y_nottreat, y_treated, y_HetTreated)}
    \hlkwd{save}\hlstd{(SimTreat,}\hlkwc{file} \hlstd{=} \hlstr{"SimTreat.Rda"}\hlstd{)}
    \hlkwd{attach}\hlstd{(SimTreat)}
\end{alltt}


{\ttfamily\noindent\itshape\color{messagecolor}{\#\# The following objects are masked \_by\_ .GlobalEnv:\\\#\# \\\#\#\ \ \ \  Edu, Gen, Inc, Treat, y\_HetTreated, y\_nottreat, y\_treated}}\end{kframe}
\end{knitrout}


% Table created by stargazer v.5.2 by Marek Hlavac, Harvard University. E-mail: hlavac at fas.harvard.edu
% Date and time: Thu, Jul 06, 2017 - 17:06:03
\begin{table}[!htbp] \centering 
  \caption{Descriptive statistics} 
  \label{} 
\begin{tabular}{@{\extracolsep{5pt}}lccccc} 
\\[-1.8ex]\hline 
\hline \\[-1.8ex] 
Statistic & \multicolumn{1}{c}{N} & \multicolumn{1}{c}{Mean} & \multicolumn{1}{c}{St. Dev.} & \multicolumn{1}{c}{Min} & \multicolumn{1}{c}{Max} \\ 
\hline \\[-1.8ex] 
Inc & 1,000 & 227.385 & 289.583 & 2.871 & 3,957.030 \\ 
y\_nottreat & 1,000 & 0.028 & 0.947 & $-$3.128 & 2.720 \\ 
y\_treated & 1,000 & 0.374 & 1.232 & $-$3.279 & 4.863 \\ 
y\_HetTreated & 1,000 & 0.469 & 1.348 & $-$3.573 & 5.946 \\ 
\hline \\[-1.8ex] 
\end{tabular} 
\end{table} 
% latex table generated in R 3.3.2 by xtable 1.8-2 package
% Thu Jul  6 17:06:03 2017
\begin{table}[ht]
\centering
\caption{Treatment assignment} 
\begin{tabular}{rrr}
  \hline
 & TreatCount & TreatProp \\ 
  \hline
Control & 499 & 49.90 \\ 
  Treatment & 501 & 50.10 \\ 
   \hline
\end{tabular}
\end{table}
% latex table generated in R 3.3.2 by xtable 1.8-2 package
% Thu Jul  6 17:06:03 2017
\begin{table}[ht]
\centering
\caption{Gender} 
\begin{tabular}{rrr}
  \hline
 & GenCount & GenProp \\ 
  \hline
Male & 476 & 47.60 \\ 
  Female & 524 & 52.40 \\ 
   \hline
\end{tabular}
\end{table}
% latex table generated in R 3.3.2 by xtable 1.8-2 package
% Thu Jul  6 17:06:03 2017
\begin{table}[ht]
\centering
\caption{Education} 
\begin{tabular}{rrr}
  \hline
 & EduCount & EduProp \\ 
  \hline
Primary school & 495 & 49.50 \\ 
  High school & 309 & 30.90 \\ 
  University \& above & 196 & 19.60 \\ 
   \hline
\end{tabular}
\end{table}


\clearpage

    \subsubsection{Regression}

\begin{knitrout}
\definecolor{shadecolor}{rgb}{0.969, 0.969, 0.969}\color{fgcolor}\begin{kframe}
\begin{alltt}
  \hlcom{#Reg with no treatment effect}
  \hlstd{NoTreReg} \hlkwb{<-} \hlkwd{lm}\hlstd{(y_nottreat} \hlopt{~} \hlstd{Treat)}
  \hlkwd{summary}\hlstd{(NoTreReg)}
\end{alltt}
\begin{verbatim}
## 
## Call:
## lm(formula = y_nottreat ~ Treat)
## 
## Residuals:
##     Min      1Q  Median      3Q     Max 
## -3.1983 -0.6181 -0.0039  0.6118  2.6498 
## 
## Coefficients:
##                Estimate Std. Error t value Pr(>|t|)  
## (Intercept)     0.07057    0.04238   1.665   0.0962 .
## TreatTreatment -0.08414    0.05988  -1.405   0.1603  
## ---
## Signif. codes:  0 '***' 0.001 '**' 0.01 '*' 0.05 '.' 0.1 ' ' 1
## 
## Residual standard error: 0.9467 on 998 degrees of freedom
## Multiple R-squared:  0.001975,	Adjusted R-squared:  0.0009747 
## F-statistic: 1.975 on 1 and 998 DF,  p-value: 0.1603
\end{verbatim}
\begin{alltt}
  \hlcom{#Reg with treatment effect}
  \hlstd{TreReg} \hlkwb{<-} \hlkwd{lm}\hlstd{(y_treated} \hlopt{~} \hlstd{Treat)}
  \hlkwd{summary}\hlstd{(TreReg)}
\end{alltt}
\begin{verbatim}
## 
## Call:
## lm(formula = y_treated ~ Treat)
## 
## Residuals:
##     Min      1Q  Median      3Q     Max 
## -3.9557 -0.7723  0.0471  0.7685  4.1863 
## 
## Coefficients:
##                Estimate Std. Error t value Pr(>|t|)    
## (Intercept)     0.07057    0.05350   1.319    0.187    
## TreatTreatment  0.60632    0.07559   8.021 2.92e-15 ***
## ---
## Signif. codes:  0 '***' 0.001 '**' 0.01 '*' 0.05 '.' 0.1 ' ' 1
## 
## Residual standard error: 1.195 on 998 degrees of freedom
## Multiple R-squared:  0.06057,	Adjusted R-squared:  0.05962 
## F-statistic: 64.34 on 1 and 998 DF,  p-value: 2.918e-15
\end{verbatim}
\begin{alltt}
  \hlcom{#Controlling for Education}
  \hlstd{TreRegEdu} \hlkwb{<-} \hlkwd{lm}\hlstd{(y_treated} \hlopt{~} \hlstd{Treat} \hlopt{+} \hlstd{Gen} \hlopt{+} \hlstd{Edu)}
  \hlkwd{summary}\hlstd{(TreRegEdu)}
\end{alltt}
\begin{verbatim}
## 
## Call:
## lm(formula = y_treated ~ Treat + Gen + Edu)
## 
## Residuals:
##     Min      1Q  Median      3Q     Max 
## -3.9731 -0.7830  0.0493  0.7539  4.1317 
## 
## Coefficients:
##                       Estimate Std. Error t value Pr(>|t|)    
## (Intercept)            0.12498    0.07598   1.645    0.100    
## TreatTreatment         0.60648    0.07575   8.006 3.28e-15 ***
## GenFemale             -0.07821    0.07578  -1.032    0.302    
## EduHigh school        -0.03720    0.08680  -0.429    0.668    
## EduUniversity & above -0.01028    0.10105  -0.102    0.919    
## ---
## Signif. codes:  0 '***' 0.001 '**' 0.01 '*' 0.05 '.' 0.1 ' ' 1
## 
## Residual standard error: 1.196 on 995 degrees of freedom
## Multiple R-squared:  0.06174,	Adjusted R-squared:  0.05797 
## F-statistic: 16.37 on 4 and 995 DF,  p-value: 5.384e-13
\end{verbatim}
\begin{alltt}
  \hlcom{#Het effects with Gen}
  \hlstd{TreRegHet} \hlkwb{<-} \hlkwd{lm}\hlstd{(y_HetTreated} \hlopt{~} \hlstd{Treat}\hlopt{*}\hlstd{Gen} \hlopt{+} \hlstd{Edu)}
  \hlkwd{summary}\hlstd{(TreRegHet)}
\end{alltt}
\begin{verbatim}
## 
## Call:
## lm(formula = y_HetTreated ~ Treat * Gen + Edu)
## 
## Residuals:
##     Min      1Q  Median      3Q     Max 
## -4.5257 -0.7842  0.0533  0.7751  4.9736 
## 
## Coefficients:
##                          Estimate Std. Error t value Pr(>|t|)    
## (Intercept)               0.07181    0.09192   0.781    0.435    
## TreatTreatment            0.70752    0.11846   5.972 3.25e-09 ***
## GenFemale                 0.01894    0.11577   0.164    0.870    
## EduHigh school           -0.02574    0.09363  -0.275    0.783    
## EduUniversity & above    -0.01995    0.10897  -0.183    0.855    
## TreatTreatment:GenFemale  0.17403    0.16346   1.065    0.287    
## ---
## Signif. codes:  0 '***' 0.001 '**' 0.01 '*' 0.05 '.' 0.1 ' ' 1
## 
## Residual standard error: 1.289 on 994 degrees of freedom
## Multiple R-squared:  0.08996,	Adjusted R-squared:  0.08538 
## F-statistic: 19.65 on 5 and 994 DF,  p-value: < 2.2e-16
\end{verbatim}
\begin{alltt}
  \hlcom{#Het effects with Gen and robust SE}
  \hlstd{vce} \hlkwb{<-} \hlkwd{vcovHC}\hlstd{(TreRegHet,} \hlkwc{type} \hlstd{=} \hlstr{"HC1"}\hlstd{)}
  \hlstd{TreRegHetRobSE} \hlkwb{<-} \hlkwd{sqrt}\hlstd{(}\hlkwd{diag}\hlstd{(vce))}
  \hlstd{TreRegHetRobF} \hlkwb{<-} \hlkwd{waldtest}\hlstd{(TreRegHet,} \hlkwc{vcov} \hlstd{= vce)}
  \hlstd{TreRegHetRobF}
\end{alltt}
\begin{verbatim}
## Wald test
## 
## Model 1: y_HetTreated ~ Treat * Gen + Edu
## Model 2: y_HetTreated ~ 1
##   Res.Df Df      F    Pr(>F)    
## 1    994                        
## 2    999 -5 19.414 < 2.2e-16 ***
## ---
## Signif. codes:  0 '***' 0.001 '**' 0.01 '*' 0.05 '.' 0.1 ' ' 1
\end{verbatim}
\end{kframe}
\end{knitrout}


% Table created by stargazer v.5.2 by Marek Hlavac, Harvard University. E-mail: hlavac at fas.harvard.edu
% Date and time: Thu, Jul 06, 2017 - 17:06:04
\begin{table}[!htbp] \centering 
  \caption{Regression results} 
  \label{} 
\begin{tabular}{@{\extracolsep{5pt}}lccccc} 
\\[-1.8ex]\hline 
\hline \\[-1.8ex] 
 & \multicolumn{5}{c}{\textit{Dependent variable:}} \\ 
\cline{2-6} 
\\[-1.8ex] & \underline{Untreated Y} & \multicolumn{2}{c}{\underline{Treated Y}} & \multicolumn{2}{c}{\underline{Het. Treated Y}} \\ 
\\[-1.8ex] & (1) & (2) & (3) & (4) & (5)\\ 
\hline \\[-1.8ex] 
 Treatment & $-$0.084 & 0.606$^{***}$ & 0.606$^{***}$ & 0.708$^{***}$ & 0.708$^{***}$ \\ 
  & (0.060) & (0.076) & (0.076) & (0.118) & (0.111) \\ 
  Female &  &  & $-$0.078 & 0.019 & 0.019 \\ 
  &  &  & (0.076) & (0.116) & (0.090) \\ 
  High school &  &  & $-$0.037 & $-$0.026 & $-$0.026 \\ 
  &  &  & (0.087) & (0.094) & (0.095) \\ 
  University &  &  & $-$0.010 & $-$0.020 & $-$0.020 \\ 
  &  &  & (0.101) & (0.109) & (0.104) \\ 
  Female $\times$ Treatment &  &  &  & 0.174 & 0.174 \\ 
  &  &  &  & (0.163) & (0.163) \\ 
  Constant & 0.071$^{*}$ & 0.071 & 0.125 & 0.072 & 0.072 \\ 
  & (0.042) & (0.054) & (0.076) & (0.092) & (0.075) \\ 
 \hline \\[-1.8ex] 
Observations & 1,000 & 1,000 & 1,000 & 1,000 & 1,000 \\ 
Adjusted R$^{2}$ & 0.001 & 0.060 & 0.058 & 0.085 & 0.085 \\ 
F Statistic & 1.975 & 64.341$^{***}$ & 16.370$^{***}$ & 19.652$^{***}$ & 19.652$^{***}$ \\ 
\hline 
\hline \\[-1.8ex] 
\multicolumn{6}{l}{\parbox[t]{10cm}{$^{*}p<0.1;^{**}p<0.05;^{***}p<0.01$. Standard errors in parentheses.}} \\ 
\end{tabular} 
\end{table} 


\end{document}
